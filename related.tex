%!TEX root = paper.tex
%related.tex

\section{Related work}
\label{sec:related}

\subsection{Privacy and Government}
This work falls within a line of research that investigates how privacy affects the balance of power between citizens and the state.  Laskowski and Johnson investigate the application of surveillance technology by a government that wishes to remain in power~\cite{laskowskigovernment}.  A major takeaway is that enhanced surveillance technology increases incentives for abuse.  In a similar vein, Goh provides a model of a government that may employ surveillance to lower the risk of a terrorist attack \cite{goh2015prosperity}.  Greater surveillance carries an increased risk that citizens will learn of its existence, which increases the risk that the government loses power.  Goh finds that a rational government will employ less surveillance when citizens value their privacy more, but autocratic governments will employ more surveillance than democratic ones.

A larger literature examines the relationship between citizens and the state in general.  Downs~\cite{downs1957economic} provides a model of political competition based on a continuum of political preferences, extending Hoteling's study of horizontal differentiation~\cite{press39hotelling}.  Further studies model the process by which governments are overthrown.  Ginkel and Smith~\cite{ginkel1999so} consider factors that determine the probability of revolution in a repressive regime.  Lohmann~\cite{lohmann1994dynamics} describes the potential overthrow of a government through an informational cascade model. Kuran~\cite{kuran1989sparks} attempts to explain why revolutions often take the world by surprise with a game theoretic model of political change.  These studies do not consider the effects of privacy in determining political outcomes.

\subsection{Privacy and Firms}

Privacy also affects the relationship between citizens and firms, and several strands of research shed light on this topic (Acquisti provides a survey \cite{acquisti2010economics}).  Privacy can be seen in the classic literature on information economics as an information imbalance between a principal and an agent.  Moral hazard models assume that an agent's actions are not directly observable and the focus is on aligning incentives through contracting \cite{holmstrom1979moral}\cite{stiglitz1981credit}.  In models of adverse selection, agent types are private and certain types are driven out of a market because the principal cannot distinguish between them \cite{akerlof1995market}.  In signaling games, agents may engage in costly actions to signal their private type for economic gain \cite{spence1973job}.  None of these settings correspond to our focus on privacy as protecting a marginalized group.  Furthermore, models in this literature are usually neoclassical in the sense that privacy is an obstacle to maximizing welfare. 

An emerging body of privacy research models the behavior of consumers that participate in two different markets in sequence.  Firms in one market learn about consumers based on their purchase decisions and may be allowed to sell this information to firms in the second market.  A common theme in this literature is that outcomes depend on whether consumers are myopic, considering each purchase decision without regard for how it will affect future purchases, or fully sophisticated.  In \cite{johnsoncaviar}, Johnson and Laskowski find that when consumers are myopic, firms benefit greatly, but consumer surplus is also reduced. When consumers are strategic, they end up better off, but firms fare worse.  In a similar vein, Acquisti and Varian look at a single monopolist that sells two goods in series \cite{acquisti2005conditioning}.  Taylor examines the case of two firms when consumer valuations for each good can take on two values, but these valuations are not perfectly correlated with each other \cite{taylor2004consumer}.  Information sharing may increase or decrease consumer surplus and welfare, depending on the demand specification.  Calzolari and Pavan apply a mechanism design framework in which firms may offer arbitrary contracts to users \cite{calzolari2006optimality}.  They describe conditions in which an upstream firm may find it optimal to offer consumers full privacy.

Other studies provide further examples of scenarios in which privacy is welfare-enhancing.  Hermalin and Katz discuss insurance markets and investments in information gathering \cite{hermalin2006privacy}.  Taylor considers a scenario in which collecting information about customers is costly and firms may overinvest in this activity \cite{taylor2003privacy}.  Hann et al. argue that unsolicited marketing imposes negative costs on consumers in the absence of privacy regulation \cite{hann2008consumer}.

In contrast to the studies we mention, which treat privacy as a binary parameter, our work distinguishes four distinct types of privacy, which are inspired by technologies and legal debate.  We further apply our model to explore the effects of privacy on divisive laws.

\subsection{Technological Privacy Protections}

Members of a marginalized group may employ a variety of technologies to enhance their privacy.  These may be divided into technologies that conceal specific behaviors, and technologies that maintain the secrecy of personal characteristics.  In the former category are technologies like virtual private networks and Tor, which allows anonymous internet browsing~\cite{dingledine2003privacy}.  Cryptocurrencies like Bitcoin allow decentralized payments while making it difficult to discover the owners of individual accounts.  Smart contracts on platforms like Ethereum allow a richer set of computations while concealing the identities of contracting parties~\cite{buterin2014next}.  Darknet markets provide a platform on which buyers and sellers of forbidden activities may connect with each other~\cite{soska2015measuring}.

A second category of technologies protects personal information found in databases.  Today's companies increasing store a rich variety of customer data, which may include browsing history, purchases, physical location, search terms, and so on.  Such data can be used to infer a variety of personal characteristics, including minority status.  Moreover, aggregating information from multiple datasets can improve the ability of a government to identify desired individuals~\cite{acquisti2009predicting}.

A number of techniques have been proposed to protect information contained in databases.  Early attempts focused on syntactic notions like k-anonymity~\cite{sweeney2002k}, which requires that a database entry corresponding to any individual appears multiple times.  Such notions involve a relatively weak adversary model and are vulnerable to known attacks.  More recently, a series of papers developed the standard of differential privacy, which requires that the probability of any given algorithm output can only change by a bounded amount if a single entry is altered~\cite{dwork2014algorithmic}.  Such a guarantee may often be achieved by adding a calibrated amount of noise to an algorithm~\cite{dwork2006calibrating}.  Alternately, the exponential mechanism of McSherry and Talwar may be used to provide privacy over discrete outcomes~\cite{mcsherry2007mechanism}.

Trusted hardware like Intel's Software Guard Extensions are designed to allow data to be processed securely, even when the underlying computer is not trusted~\cite{anati2013innovative}.  Cheng, et al. present a system that combines trusted hardware with a blockchain to enable differentially private computations in a distributed network~\cite{cheng2018ekiden}.

An ongoing debate surrounds the use of privacy-enhancing technologies, in contexts ranging from ethics\cite{lyon2002surveillance,diffie2010privacy}, to law\cite{landau2009,landau2013making,landau2014,Bankston14}, to security-relevant effectiveness~\cite{schneier2013oppression%,schneier2013surveillance
}.  While our model abstracts from these details, we will use it to explore the impact of technologically-based privacy.


\subsection{Polarization}
Most voters in the United States are overwhelmingly moderate in their policy positions\cite{layman2006party}.   Nevertheless, the United States Congress has passed a number of divisive laws, many of which have been challenged and overturned by the US Supreme Court.  Divisive laws have also been passed in European countries, such as those against face covering, pejoratively dubbed ``burka bans.''


\begin{figure}[htbp]
\begin{center}
\includegraphics[width=0.49\textwidth]{figs/polar_house_and_senate_46-115_july_11}
\caption{{\bf Distance between Parties in the US House and Senate, 1879-2015}}
\label{fig:uscongress}
\end{center}
\end{figure}

The United States congress in particular has become increasingly polarized over the last 40 years (see Figure \ref{fig:uscongress}).  
%Today, members of congress exhibit a distinctly bimodal distribution in terms of political preferences, as seen in \ref{fig:partisonship}.  
Researchers have posited a number of reasons for this phenomenon \cite{barber2015causes}\cite{poole1984polarization}, ranging from a polarized electorate, to southern realignment, to gerrymandering, to the evolution of modern primary elections, to economic inequality, to money in politics, to the media environment, or to congress-based factors such as congressional rule changes, majority party agenda control, party pressures, teamsmanship, or the breakdown of bipartisan norms.  All of these issues are discussed in~\cite{poole1984polarization}.
More culturally-specific theories involving authoritarianism are also prevalent \cite{hetherington2009authoritarianism}.

%\begin{figure}[htbp]
%\begin{center}
%\includegraphics[width=0.4\textwidth]{figs/alpha_House_114_Histogram_8_January_2016}
%\caption{{\bf Partisonship in the US House of Representatives}}
%\label{fig:partisonship}
%\end{center}
%\end{figure}


%; and there is a remarkably close correlation between economic
%inequality and polarization in the United States~\cite{mccarty2006polarized}.  See Figure \ref{fig:inequality}.

%\begin{figure}[htbp]
%\begin{center}
%\includegraphics[width=0.4\textwidth]{figs/8141eb7a0}
%\caption{{\bf Polarization versus Income Inequality}}
%\label{fig:inequality}
%\end{center}
%\end{figure}


A more economically-driven explanation derives from the notion of information cascades. An information cascade occurs when people receive a noisy informational signal and observe the behavior of friends and colleagues to inform decision-making.  Although agents are individually rational, they may find it optimal to rely on the information they derive from previous agents, ignoring their private signals \cite{bikhchandani1992theory}.

The notion that people exhibit herding behavior in predictable circumstances has been around for decades \cite{shiller1995conversation}.  For example, researchers at Iowa State University conducted 259 interviews with farmers who had largely refused offers to adopt drought-resistant seed corn during the Great Depression and Dust Bowl.  They found that the slow rate of adoption was due to ``how farmers valued the opinion of their friends and neighbors instead of the word of a salesman''\cite{beal1957diffusion}.


We include a notion of influenced behavior in our model as a way to describe the polarization of society.  Our model does not mandate that citizens consider how a law affects other citizens, but merely allows it.  %This allowance appears justified by the observation that the United States has a legislative system that is organized like Figure \ref{fig:partisonship}, while it has an electorate that is more like Figure \ref{fig:voters}.




%
%\begin{figure}[htbp]
%\begin{center}
%\includegraphics[width=0.4\textwidth]{figs/polarization2}
%\caption{{\bf Long Term Trends in US Voter Ideology}}
%\label{fig:voters}
%\end{center}
%\end{figure}
%




