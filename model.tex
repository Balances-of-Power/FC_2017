%!TEX root = paper.tex
%model.tex

\section{Model}
\label{sec:model} 


%\begin{tikzpicture}[scale = 6]
%
%       % x axis
%\draw [->] (0, 0) -- (1.1, 0) node [below right] {$P_0$};
%\draw [shift={(1.0, 0)}] (0,0.02) -- (0, -0.02) node[below] {$Cr(0)$};
%
%       % y axis
%\draw [->] (0, 0) -- (0, 1.3) node [above left] {$P_1$};
%\draw [shift={(0, 1.2)}] (-0.02,0) -- (0.02,0) node[shift={(-0.8,0)}] {$Cr(1)$};
%
%	%diagonal
%\draw [->,dashed] (-0.1, -0.1) -- (1.2, 1.2) node [above] {$B$};
%
%	%right
%\draw [-,dashed] (1, 0) -- (1, 1.2);
%
%	%top
%\draw [-,dashed] (0, 1.2) -- (1, 1.2); 
%
%	%point A
%\draw[shift={(0.2, 1.2)}] node [above] {$A$};
%
%\end{tikzpicture}

%
%\usetikzlibrary{patterns}
%
%\newcommand{\myaxes}{
%	\draw [->|] (0,0) node [left=1mm] {\scriptsize $$} -- node [rotate=90,above] {vertical label} (0,1.2) node [left=1mm,overlay] {\scriptsize $Cr(1)$};
%	\draw [->|] (0,0) node [below=1mm] {\scriptsize $0$} -- node [below] {horizontal label} (1,0) node [below=1mm] {\scriptsize $Cr(0)$};
%}
%
%\tikzstyle{full-protection}=[blue,pattern=north east lines,pattern color=blue];
%\tikzstyle{full-self}=[orange,pattern=north west lines,pattern color=orange];
%\tikzstyle{full-market}=[red,pattern=crosshatch dots,pattern color=red];
%
%\tikzstyle{partial-protection}=[blue,pattern=north east lines,pattern color=blue,opacity=.4];
%\tikzstyle{partial-self}=[orange,pattern=north west lines,pattern color=orange,opacity=.4];
%\tikzstyle{partial-market}=[red,pattern=crosshatch dots,pattern color=red,opacity=.4];
%
%\tikzstyle{partial-all}=[gray,pattern=grid,pattern color=gray,opacity=.4];
%
%\tikzstyle{passivity}=[dashed,thick];
%\tikzstyle{special}=[red,thick];
%
%\begin{tikzpicture}
%
%	%	Legend
%	
%%	\draw [special] (0,-.5) rectangle ++(.25,.25) ++(0,-.125) node [right=2mm,black,opacity=1] {partial protection and full market insurance};
%
%	\draw [partial-self,line width=.8mm,dash pattern=on 1mm off 1mm] (-.2,-.875)--++(.6,0) node [right=0mm,black,opacity=1] {area 6};
%	\draw [partial-market,line width=.8mm,dash pattern=on 1mm off 1mm] (.4,-.875)--++(-.6,0);
%
%	\draw [partial-protection,line width=.6mm] (-.2,-1.375)--++(.6,0) node [right=0mm,black,opacity=1] {area 5};
%	
%	\draw [partial-all] (0,-.5) rectangle ++(.25,.25) ++(0,-.125) node [right=2mm,black,opacity=1] {area 4};
%	\draw [full-market] (0,0) rectangle ++(.25,.25) ++(0,-.125) node [right=2mm,black,opacity=1] {area 3};
%	\draw [full-self] (0,.5) rectangle ++(.25,.25) ++(0,-.125) node [right=2mm,black,opacity=1] {area 2};
%	\draw [full-protection] (0,1) rectangle ++(.25,.25) ++(0,-.125) node [right=2mm,black,opacity=1] {area 1};
%\end{tikzpicture}
%
%\begin{tikzpicture}[>=stealth,scale=7]
%
%%	\draw [partial-all,draw=none] (.125,0) rectangle (1,.5);
%%	\draw [full-self,draw=none] (0,0)--(0,.375)--(.625,.375)--(1,0)--(0,0);
%%	\draw [full-protection,draw=none] (0,0) -- (.125,.125) -- (.125,1) -- (0,1) -- (0,0);
%%	% \draw [full-market,draw=none] (.125,.5) rectangle (1,1);
%%	\draw [full-market,draw=none] (0,.5) rectangle (1,1); % combined (1) and (5)
%%	
%%	\draw [partial-market,line width=.8mm,dash pattern=on 1mm off 1mm] (.5,.5) node [below,black] {\small FIX ME}--(1,.5);
%%	\draw [partial-self,line width=.8mm,dash pattern=on 1mm off 1mm] (1,.5)--(.5,.5);
%%	
%%	\draw [partial-protection,line width=.6mm] (.125,.5)--(.125,1);
%	
%%	\draw [special] (0,0) -- (.5,.5) -- (0,.5) -- (0,0);
%	\myaxes{}
%\end{tikzpicture}


\subsection{Definitions}
As we use the terms divisiveness, polarization, and privacy throughout the paper, we take care to define them here.

\begin{itemize}
\item  \emph{Divisiveness} of a law refers to the extent to which the different citizens value the law differently -- measured, for example, by the standard deviation in valuations.

\item \emph{Polarization} of a society with respect to a law refers to the grouping of individuals into sets having disparate valuation of that law. %Our generic model witnesses polarization directly through its influence parameters; whereas in practice we can usually only measure the polarization of outcomes, leaving the incentive mechanism opaque. %which may or may not reflect the polarization of individual actors.  

\item \emph{Privacy} refers to an individual's ability to possess property (physical or intellectual) that is free from inspection or search. %This paper will focus on laws related to privacy in this sense.   %substantial portion of our  Many of our laws Privacy is a right afforded by most democratic societies primarily motivated by preventing abuse.

\end{itemize}

\subsection{Overview}
Our general model presents an incentive structure for explaining polarization of a society with respect to divisive laws. Given a specific law, our model assumes the following. 
\begin{itemize}
\item Each individual begins with an initial evaluation of the law. We refer to this initial evaluation as the derived benefit for that individual.
\item Each individual's support for a law may (but need not) be influenced by other individuals.
\item Individuals are heterogeneous in their ability to influence others.
%\item Each individual may have a non-commutative influence on another individual's evaluation.
%%\item We distinguish an individual's derived benefit of a law from an individual's support of a law. 
%\item An individual's support is a function of their derived benefit together with a weighted influence of others. %influence With respect to a fixed law, suppose the mixing process happens and we measure, then we cannot tell the difference between the influence and the case where each individual makes their own evaluation or is influenced by others.  
\end{itemize}

%Want to talk about privacy.
\subsection {Citizen Influence}
For each pair of citizens $$i,j\in\{1,\dots N\},$$ define $$a_{ij}\in [0,1]$$ to be the influence of person $j$ on person $i$.  We may think of $a_{ij}$ also as the affinity person $i$ has toward person $j$. 

% into account when making decisions, or the extent to which person $j$ influences the decision of person $i$. 

Each individual will arrive at a level of support for the law based on a weighted average of the valuations of their influences, where the weighting is determined by the influence parameters $a_{ij}$. To be consistent with the notion of weighted average, we thus require $$\sum_ja_{ij}=1.$$%something similarly concrete-ish.

\subsection{Citizen Valuations of Laws}
Given a law $L$ and a citizen $j\in\{1,\dots N\}$, %define $$L(j)\in[0,1]$$ to be the probability that the law targets $j$.
%Alternatively, 
let $$V_L(i)\in\mathbb{R}$$ represent the direct impact of the law on person $i$. We assume that the direct impact of $L$ is unbounded because laws may save a life or take away life from individuals impacted by them.

If an individual's direct valuation for a law is perfectly in tune with the direct impact on the law then this person's support for a law will be $V_L(i)$. % weight his or her own opinion so much above others that for all practical purposes we have $a_{ii}\approx 1$, $a_{ij}\approx 0$ for $j\neq i$, and the support of $i$ for $L$ is exactly $$a_{ii}L(i)=L(i).$$
In other cases, for example where a law has little or no direct impact on the individual, a ban on muslims for example, initial evaluation of the law may be something very close to zero, i.e. neither positive nor negative.  Nevertheless a person's view of the law may evolve based on the views and experiences of their influences.  In such cases, we may suppose that after some amount of time, an individual's support for L evolves to become $$U_L(i)=\sum_ja_{ij}V_L(j).$$


%If we order the elements of this vector and graph it on a line, we can see directly from the graph a number of things
%\begin{enumerate}
%\item what percentage supports the law -- aka where the graph crosses the horizontal axis
%\item the existence of distinct homogeneous clusters -- regions along the horizontal axis where the function takes on a homogeneous value
%\item the extent to which the law is divisive for the society -- aka the magnitude of disutility for non-supporters of the law.
%\end{enumerate}
%
%How does privacy enter this picture?
%Privacy decreases the law's overall enforceability, and in particular, it makes it harder to enforce the law selectively.
%
%The crux of the analysis will be to show that privacy is especially welfare-enhancing in cases where a society is more polarized.