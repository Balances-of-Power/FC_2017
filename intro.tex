%!TEX root = paper.tex
%intro.tex
%

\section{Introduction}
\label{sec:intro}




In 2006, Utah police received an anonymous tip about drugs being sold out of a house in South Salt Lake. Detective Douglas Fackrell spent several hours surveilling the house in an unmarked car, but saw only modestly suspicious activity.  After a week, he stopped an individual, Edward Strieff, as he was exiting the house and asked for identification.  When he discovered that Strieff had an outstanding warrant for a minor traffic violation, Detective Fackrell proceeded to search him and found methamphetamine in Strieff's pockets.  

This event marked the start of legal battle that reached the US Supreme Court this year.  Although a warrant generally provides adequate reason for a search, the police conceded that Fackrell did not have reasonable suspicion to detain Strieff in the first place.  The supreme court has long held that evidence gained in violation of the Fourth Amendment is inadmissible as evidence - the so-called exclusionary rule.  The case revolves around the extent to which the exclusionary rule protects the privacy of citizens when new information, such as an arrest warrant, arrives after an improper detainment.

Utah v. Strieff reveals the extent to which our value of privacy is bound up in notions of power and polarization.  During oral argument in February, Justice Sonya Sotomayor posed the following question. 

\begin{quote}``What stops us from becoming a police state and just having the police stand on the corner down here and stop every person, ask them for identification, put it through, and if a warrant comes up, searching them?"
\end{quote}

At the heart of this argument is a recognition that privacy protects entire groups of people.  One doesn't have to have drugs in one's pocket to object to arbitrary searches by police.  Privacy places limitations on police power, affecting the playing field faced by all citizens.  Even when the vast majority of police officers abide by strict ethical standards, the prospect of running into a corrupt one remains threatening.  Furthermore, the ability to invade the privacy of citizens has been argued to increase incentives for governments to abuse their power.~\cite{laskowskigovernment}

Calls for privacy are particularly acute when a particular group is disadvantaged or marginalized.  In this case, the laws that police enforce may disproportionally affect the marginalized group.  A classic example is the disparity between sentences for powder cocaine, typically associated with rich white communities, and crack cocaine, which is associated with disadvantaged black communities.  Before the fair sentencing act of 2010, the weight of powder cocaine needed to trigger certain federal criminal penalties was 100 times greater than the weight of crack cocaine that would trigger the same penalties.  This disparity is said to be a significant factor behind the large number of African Americans that were sentenced for drug offenses.~\cite{beaver2009getting}

To understand how laws and police enforcement affect disadvantaged groups, we must also understand how society is polarized between different groups of people to begin with.  How are laws passed that benefit one group at the expense of another group.  Moreover, can privacy protection help marginalized groups overcome their disadvantageous position?

In this paper, we use game theoretic modeling to explore the connections between privacy, polarization, and the passage of divisive laws.  Our framework is based on a population of citizens that influence what laws are passed, or what laws are maintained.  A law is defined in terms of how it impacts each individual, and our model is flexible in that it allows any set of effects.  We define a notion of divisiveness which allows us to measure the extent to which a law disproportionately affects different groups of citizens.

Divisiveness is not the only factor to consider when evaluating laws.  A divisive law may still be justified if it significantly improves welfare.  Progressive taxation is one example in which a law targets groups differently with the frequent aim of enhancing welfare.  On the other hand, some laws may not be divisive at all, but may still be welfare-decreasing or unjust for other reasons.  Nevertheless, we believe that divisiveness should generally be viewed as a cause for concern, especially when a law targets a marginalized group.

Our framework allows citizens to form opinions based on how a law impacts them directly, but it optionally allows them to consider the impact on others as well.  This is achieved through an influence matrix that is multiplied by the direct effect of the law.  This can be used to represent a concern for friends, loyalty to a larger group, or learning from a small number of influential personalities.  The influence matrix also allows us to discuss how polarized society is.  At the end of our analysis, we construct a matrix to model the case of a society with one majority group and one minority group.

Our model assumes that laws that are supported by a majority of citizens are passed or maintained.  Although the democratic process involves many factors before gathering a simple majority, we believe this is a useful and tractable way to explain the types of laws that exist in society.

Using our model of how laws are enforced, we are able to identity four distinct notions of privacy.  Two of these are technological, including strategies that citizens can take to hide features and behaviors from authorities.  The other two are legal notions, depending on a judicial branch that functions as a check on enforcement procedures. %police behavior. 
We describe the function of each of these privacy notions using our two-population model of society.  We find that each type of privacy allows a different set of laws to be passed and enforced, resulting in different effects on divisiveness.  Our work supports the idea that privacy, while far from a perfect cure, has a role to play in mitigating the divisive effects of laws in a polarized society.

%Discussion examples
%\begin{itemize}
%\item Fergeuson style searches
%\item stop and frisk
%\item prohibition
%\item iphone data
%\item Muslim databases
%\end{itemize}


