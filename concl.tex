%!TEX root = paper.tex
%concl.tex

\section{Discussion and Conclusion}
\label{sec:concl}

Our study is an attempt to understand privacy, not at the level of individual incentives, but at the level of communities and the relationships they have to each other and to the state.  Though our modeling framework is exploratory, it reveals some of the complexity inherent in these relationships.  We were able to relate privacy to polarization and the divisiveness of laws, but found that outcomes depend critically on what notion of privacy is in play.  Privacy enforced through technology can have dramatic effects on what laws are enforced, but all laws may be rendered ineffective whether divisive or not.  A privacy technology may work, after all, whether it is concealing sexual behavior between consenting adults, or a plot to assassinate a leader in government.  Legal notions of privacy hold the promise of more precise judgements.  Courts can identify disadvantaged groups and extend protections to them, without extending those same protections to say, serial killers.  Yet our model predicts that here too, privacy is not perfectly tailored to prevent the enforcement of divisive laws.  Further research is needed to assess how privacy laws may work in conjunction with anti-discrimination laws and policies, to better protect marginalized groups. 

In the future, we plan to extend our model to capture more features of the legal system and of privacy protection.  One important addition will be a description of how authorities gather information on potential law breakers.  A detailed model might describe citizens with a collection of attributes, each of which may carry information about a citizen's propensity to violate a specific law.  Some attributes may be hidden with technological privacy measures, some may be the topic of legal protection, and others may remain public and available to police as they direct their investigations.  

We would further like to model the issues that arise when police have the right to search a large number of people, but only enough resources to choose a small number.  The ability to arbitrarily select which citizens to search carries a significant amount of power, even when the number of searches remains small.  A more complete model would separate legality from the actual performance of a search in order to highlight these issues.

As technology advances, many established notions of privacy face considerable pressure to evolve.  Location monitoring, deep packet inspection, linking of consumer databases, and face recognition are just a few of the threats to our ability to control our personal information.  We hope that studies like ours will help spur discussion about the role privacy plays in maintaining balances of power, and how future definitions of privacy may best be structured to protect vulnerable groups and individuals.

